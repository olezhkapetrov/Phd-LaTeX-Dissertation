{\actuality}
В колл-центрах и офисах продаж крупных компаний работают тысячи сотрудников, которые обрабатывают сотни тысяч обращений каждый день. В целях контроля качества обслуживания все разговоры записываются, формируя большие объемы аудиоданных, которые хранятся в постоянно пополняющихся архивах --- специальных базах медиаданных. Фонограммы сопровождаются метаданными: информацией об операторе, клиенте, времени и длительности звонка. Согласно внутренним регламентам организаций, данные могут удаляться по истечению какого-то срока.

Речевая аналитика фонограмм обращения клиентов позволяет извлекать критически важную для бизнеса информацию:
\begin{itemize}[beginpenalty=10000] % https://tex.stackexchange.com/a/476052/104425
  \item получать отзывы о предоставляемых товарах и услугах;
  \item оценивать качество проведенных рекламных акций;
  \item разбираться в причинах отказа от услуг;
  \item повышать качество обслуживания.
\end{itemize}

Прослушать такой объем фонограмм практически невозможно. До появления инструментов автоматической обработки аудиоданных анализ проводился вручную, на случайно выбранном подмножестве фонограмм. Такой подход позволял обрабатывать до 10\% всех данных, оставаясь при этом трудоемким и слабовоспроизводимым в силу человеческого фактора. Важная с точки зрения контроля качества информация могла оставаться за рамками выбранных для прослушивания записей.

Современные системы автоматического распознавания речи позволяют получать текстовое представление диалога клиента и оператора. Это позволяет с одной стороны исключить необходимость прослушивания фонограмм, а с другой дает возможность обрабатывать до 100\% всех данных. Контроль качества обслуживания может сводиться к формированию каскада регулярно выполняемых поисковых запросов к базе аудиоданных. Такие запросы могут включать в себя:
\begin{itemize}[beginpenalty=10000] % https://tex.stackexchange.com/a/476052/104425
  \item слова и словосочетания, обязательные или не обязательные;
  \item слова и словосочетания, обязательные для исключения из результатов поиска;
  \item временные границы слов, длительность между словами;
  \item принадлежность слов и словосочетаний одному из участников диалога --- клиенту или оператору;
  \item параметры речевой активности: длительность максимального молчания, длительность общего молчания,
 соотношение длительностей речи участников диалога и прочие временные характеристики.
\end{itemize}

Одним из способов организации поискового индекса по текстам диалогов является использование обратного индекса слов \cite{InvertedFiles}. Существует целый ряд инструментов реализующих данный подход, самым известным из которых является Lucene и построенные на его основе системы SOLR и ElasticSearch. Применительно к задачам речевой аналитики такой подход позволяет создавать решения, обеспечивающие одновременно и высокую скорость индексации, и быстрый гибкий поиск, использующий специальный формат поисковых запросов. Точность поиска при этом ограничивается точностью используемой системы автоматического распознавания речи. Для реального применения пословная ошибка распознавания диалогов не должна превышать 25--30\% WER.

На сегодняшний день существуют системы речевой аналитики для колл-центров для популярных языков, таких как русский, английский или испанский. Это возможно в том числе благодаря тому, что точность распознавания речи в телефонном канале для этих языков достаточно высокая.

Использование текстового представления диалогов остается невозможным в тех случаях, когда точность распознавания речи недостаточно высокая:
\begin{enumerate}[beginpenalty=10000] % https://tex.stackexchange.com/a/476052/104425
  \item Малоресурсные языки. Не для всех языков существует достаточно данных для обучения качественных моделей автоматического распознавания спонтанной речи даже для телефонного канала. Несмотря на то, что ведутся исследования по робастному, устойчивому к помехам, распознаванию речи для малоресурсных языков, высокий уровень ошибок делает невозможным использование поиска по текстовому представлению диалогов в задачах речевой аналитики.
  \item Удаленный микрофон. Реализация систем распознавания речи в офисах продаж на сегодняшний день является актуальным вопросом для исследователей. Низкая точность распознавания обусловлена сложной акустической обстановкой: перебивания, пересекающаяся речь, недетерминированное количество дикторов, фоновая речь и посторонние шум. Кроме того, дополнительно возникает задача диаризации --- разделения дикторов в канале. Разметка по дикторам может быть дополнительно представлена как вероятность отнесения слова к тому или иному диктору. В таких сценариях ошибка распознавания может быть больше 40\% WER.
\end{enumerate}

Поиск речевой информации по сетевому индексу фонограммы, содержащему полный набор акустических гипотез, сгенерированных декодером в процессе распознавания речи, позволяет значительно повысить полноту результата за счет возможности обнаружения слов, распознанных с низким уровнем достоверности и не попавших в текстовый результат распознавания, но сохранившихся в словной сети декодера \cite{MohriWfstAsr,WfstDecAnatomy}.

Процесс декодирования в значительной мере зависит от архитектуры системы распознавания речи. В классических гибридных системах используется WFST-декодер вместе с HCLG-каскадом, связывающем постериорные вероятности, полученные из акустической модели, с N-граммной языковой моделью. Вместе с этим в последние годы развитие получили системы основанные на сквозных архитектурах, процесс декодирования в которых отличается. Критически важным является выбор такой структуры поискового индекса, которая могла бы использоваться вместе с современными моделями распознавания речи.

Словные сети, получаемые в процессе декодирования, представляют собой связанные направленные графы без циклов, на переходах которых слова, а на узлах --- состояния. Такие сети имеют произвольную структуру переходов, которая плохо поддается индексации. В качестве более компактного представления могут использоваться другие структуры, например, сети спутывания или N-граммы \cite{ConfNetConsensus,ManguConfNetIndex,Pan2007analytical}. Существенным аспектом является сохранение временных меток словных гипотез, важных для задач речевой аналитики.

При переходе к работе с постоянно увеличивающемися объемами речевых данных, необходимо применять методы объединения сетевых индексов отдельных фонограмм в некоторые глобальные структуры --- поисковые индексы, обеспечивающие возможность сублинейного времени поиска по речевым данным.

Актуальность темы исследования подтверждается большим количеством посвященных ей докладов на международных конференциях, таких как Interspeech, ICASSP и SPECOM, а также повсеместным внедрением инструментов речевой аналитики на основе систем автоматического распознавания речи.

Учитывая вышесказанное, можно сделать вывод о необходимости разработки методов, алгоритмов и программных средств, обеспечивающих возможность индексации речевых данных, учитывающей все словные гипотезы, полученные в процессе декодирования, а также возможность полнотекстового поиска по такому индексу.

% \ifsynopsis
%     Этот абзац появляется только в~автореферате. Для формирования блоков, которые будут обрабатываться только в~автореферате,заведена проверка условия \verb!\!\verb!ifsynopsis!. Значение условия задаётся в~основном файле документа (\verb!synopsis.tex! для автореферата).
% \else
%     Этот абзац появляется только в~диссертации. Через проверку условия \verb!\!\verb!ifsynopsis!, задаваемого в~основном файле документа (\verb!dissertation.tex! для диссертации), можно сделать новую команду, обеспечивающую появление цитаты в~диссертации, но~не~в~автореферате.
% \fi

% {\progress}
% Этот раздел должен быть отдельным структурным элементом по ГОСТ, но он, как правило, включается в описание актуальности темы. Нужен он отдельным структурынм элемементом или нет --- лучше смотреть другие диссертации совета, скорее всего не нужен.

{\aim} данной диссертационной работы является исследование и разработка алгоритмов и программных средств полнотекстового поиска по речевым данным с высокой производительностью.

Для~достижения поставленной цели необходимо было решить следующие {\tasks}:
\begin{enumerate}[beginpenalty=10000] % https://tex.stackexchange.com/a/476052/104425
  \item Обзор существующих алгоритмов и программных средств для различных сценариев полнотекстового поиска на основе систем автоматического распознавания речи.
  \item Исследование алгоритмов построения поисковых индексов для речевых данных и механизмов формирования поисковых запросов.
  \item Разработка алгоритмов полнотекстового поиска на больших объемах речевых данных.
  \item Разработка программных средств, реализующих разработанные алгоритмы.
  \item Экспериментальное исследование разработанных алгоритмов и программных средств.
\end{enumerate}

{\novelty}
\begin{enumerate}[beginpenalty=10000] % https://tex.stackexchange.com/a/476052/104425
  \item Впервые \ldots
  \item Впервые \ldots
  \item Было выполнено оригинальное исследование \ldots
\end{enumerate}

{\influence} \ldots

{\methods} \ldots

{\defpositions}
\begin{enumerate}[beginpenalty=10000] % https://tex.stackexchange.com/a/476052/104425
  \item Первое положение
  \item Второе положение
  \item Третье положение
  \item Четвертое положение
\end{enumerate}

% В папке Documents можно ознакомиться с решением совета из Томского~ГУ (в~файле \verb+Def_positions.pdf+), где обоснованно даются рекомендации по~формулировкам защищаемых положений.

{\reliability} полученных результатов обеспечивается \ldots \ Результаты находятся в соответствии с результатами, полученными другими авторами.

{\probation}
Основные результаты работы докладывались~на: перечисление основных конференций, симпозиумов и~т.\:п.

{\contribution} Автор принимал активное участие \ldots

\ifnumequal{\value{bibliosel}}{0}
{%%% Встроенная реализация с загрузкой файла через движок bibtex8. (При желании, внутри можно использовать обычные ссылки, наподобие `\cite{scopus_rvect,scopus_voices}`).
    {\publications} Основные результаты по теме диссертации изложены
    в~XX~печатных изданиях,
    X из которых изданы в журналах, рекомендованных ВАК,
    X "--- в тезисах докладов.
}%
{%%% Реализация пакетом biblatex через движок biber
    \begin{refsection}[bl-author, bl-registered]
        % Это refsection=1.
        % Процитированные здесь работы:
        %  * подсчитываются, для автоматического составления фразы "Основные результаты ..."
        %  * попадают в авторскую библиографию, при usefootcite==0 и стиле `\insertbiblioauthor` или `\insertbiblioauthorgrouped`
        %  * нумеруются там в зависимости от порядка команд `\printbibliography` в этом разделе.
        %  * при использовании `\insertbiblioauthorgrouped`, порядок команд `\printbibliography` в нём должен быть тем же (см. biblio/biblatex.tex)
        %
        % Невидимый библиографический список для подсчёта количества публикаций:
        \printbibliography[heading=nobibheading, section=1, env=countauthorvak,          keyword=biblioauthorvak]%
        \printbibliography[heading=nobibheading, section=1, env=countauthorwos,          keyword=biblioauthorwos]%
        \printbibliography[heading=nobibheading, section=1, env=countauthorscopus,       keyword=biblioauthorscopus]%
        \printbibliography[heading=nobibheading, section=1, env=countauthorconf,         keyword=biblioauthorconf]%
        \printbibliography[heading=nobibheading, section=1, env=countauthorother,        keyword=biblioauthorother]%
        \printbibliography[heading=nobibheading, section=1, env=countregistered,         keyword=biblioregistered]%
        \printbibliography[heading=nobibheading, section=1, env=countauthorpatent,       keyword=biblioauthorpatent]%
        \printbibliography[heading=nobibheading, section=1, env=countauthorprogram,      keyword=biblioauthorprogram]%
        \printbibliography[heading=nobibheading, section=1, env=countauthor,             keyword=biblioauthor]%
        \printbibliography[heading=nobibheading, section=1, env=countauthorvakscopuswos, filter=vakscopuswos]%
        \printbibliography[heading=nobibheading, section=1, env=countauthorscopuswos,    filter=scopuswos]%
        %
        \nocite{*}%
        %
        {\publications} Основные результаты по теме диссертации изложены в~\arabic{citeauthor}~печатных изданиях,
        \arabic{citeauthorvak} из которых изданы в журналах, рекомендованных ВАК\sloppy%
        \ifnum \value{citeauthorscopuswos}>0%
            , \arabic{citeauthorscopuswos} "--- в~периодических научных журналах, индексируемых Web of~Science и Scopus\sloppy%
        \fi%
        \ifnum \value{citeauthorconf}>0%
            , \arabic{citeauthorconf} "--- в~тезисах докладов.
        \else%
            .
        \fi%
        \ifnum \value{citeregistered}=1%
            \ifnum \value{citeauthorpatent}=1%
                Зарегистрирован \arabic{citeauthorpatent} патент.
            \fi%
            \ifnum \value{citeauthorprogram}=1%
                Зарегистрирована \arabic{citeauthorprogram} программа для ЭВМ.
            \fi%
        \fi%
        \ifnum \value{citeregistered}>1%
            Зарегистрированы\ %
            \ifnum \value{citeauthorpatent}>0%
            \formbytotal{citeauthorpatent}{патент}{}{а}{}\sloppy%
            \ifnum \value{citeauthorprogram}=0 . \else \ и~\fi%
            \fi%
            \ifnum \value{citeauthorprogram}>0%
            \formbytotal{citeauthorprogram}{программ}{а}{ы}{} для ЭВМ.
            \fi%
        \fi%
        % К публикациям, в которых излагаются основные научные результаты диссертации на соискание учёной
        % степени, в рецензируемых изданиях приравниваются патенты на изобретения, патенты (свидетельства) на
        % полезную модель, патенты на промышленный образец, патенты на селекционные достижения, свидетельства
        % на программу для электронных вычислительных машин, базу данных, топологию интегральных микросхем,
        % зарегистрированные в установленном порядке.(в ред. Постановления Правительства РФ от 21.04.2016 N 335)
    \end{refsection}%
    \begin{refsection}[bl-author, bl-registered]
        % Это refsection=2.
        % Процитированные здесь работы:
        %  * попадают в авторскую библиографию, при usefootcite==0 и стиле `\insertbiblioauthorimportant`.
        %  * ни на что не влияют в противном случае
        \nocite{scopus_rvect}
        \nocite{scopus_voices}
        \nocite{scopus_arch}
        \nocite{rinz_globalindex}
        \nocite{rinz_asrgpu}
        \nocite{rinz_asrflow}
        \nocite{rinz_vectsearch}
        \nocite{rinz_sasearch}
    \end{refsection}%
        %
        % Всё, что вне этих двух refsection, это refsection=0,
        %  * для диссертации - это нормальные ссылки, попадающие в обычную библиографию
        %  * для автореферата:
        %     * при usefootcite==0, ссылка корректно сработает только для источника из `external.bib`. Для своих работ --- напечатает "[0]" (и даже Warning не вылезет).
        %     * при usefootcite==1, ссылка сработает нормально. В авторской библиографии будут только процитированные в refsection=0 работы.
}

% При использовании пакета \verb!biblatex! будут подсчитаны все работы, добавленные в файл \verb!biblio/author.bib!. Для правильного подсчёта работ в~различных системах цитирования требуется использовать поля:
% \begin{itemize}
%   \item \texttt{authorvak} если публикация индексирована ВАК,
%   \item \texttt{authorscopus} если публикация индексирована Scopus,
%   \item \texttt{authorwos} если публикация индексирована Web of Science,
%   \item \texttt{authorconf} для докладов конференций,
%   \item \texttt{authorpatent} для патентов,
%   \item \texttt{authorprogram} для зарегистрированных программ для ЭВМ,
%   \item \texttt{authorother} для других публикаций.
% \end{itemize}
% Для подсчёта используются счётчики:
% \begin{itemize}
%   \item \texttt{citeauthorvak} для работ, индексируемых ВАК,
%   \item \texttt{citeauthorscopus} для работ, индексируемых Scopus,
%   \item \texttt{citeauthorwos} для работ, индексируемых Web of Science,
%   \item \texttt{citeauthorvakscopuswos} для работ, индексируемых одной из трёх баз,
%   \item \texttt{citeauthorscopuswos} для работ, индексируемых Scopus или Web of~Science,
%   \item \texttt{citeauthorconf} для докладов на конференциях,
%   \item \texttt{citeauthorother} для остальных работ,
%   \item \texttt{citeauthorpatent} для патентов,
%   \item \texttt{citeauthorprogram} для зарегистрированных программ для ЭВМ,
%   \item \texttt{citeauthor} для суммарного количества работ.
% \end{itemize}

% Счётчик \texttt{citeexternal} используется для подсчёта процитированных публикаций;
% \texttt{citeregistered} "--- для подсчёта суммарного количества патентов и программ для ЭВМ.

% Для добавления в список публикаций автора работ, которые не были процитированы в автореферате, требуется их~перечислить с использованием команды \verb!\nocite! в \verb!Synopsis/content.tex!.
